\chapter{\textit{There's No One Left}}
\lettrine{W}{hen} Mary Lennox was sent to Misselthwaite Manor to live with her uncle everybody said she was the most disagreeable-looking child ever seen. It was true, too. She had a little thin face and a little thin body, thin light hair and a sour\footnote{\textbf{sour} n. unpleasant or unfriendly} expression. Her hair was yellow, and her face was yellow because she had been born in India and had always been ill in one way or another. Her father had held a position under the English Government and had always been busy and ill himself, and her mother had been a great beauty who cared only to go to parties and amuse\footnote{\textbf{amuse} v. to make someone laugh or smile} herself with gay people. She had not wanted a little girl at all, and when Mary was born she handed her over to the care of an Ayah\footnote{\textbf{Ayah} n. a domestic servant}, who was made to understand that if she wished to please the Mem Sahib she must keep the child out of sight as much as possible. So when she was a sickly, fretful\footnote{\textbf{fretful} adj. upset and worried}, ugly little baby she was kept out of the way, and when she became a sickly, fretful, toddling\footnote{\textbf{toddle} v. \textit{(of a young child)} to walk with short, unsteady steps} thing she was kept out of the way also. She never remembered seeing familiarly anything but the dark faces of her Ayah and the other native servants, and as they always obeyed her and gave her her own way in everything, because the Mem Sahib would be angry if she was disturbed by her crying, by the time she was six years old she was tyrannical\footnote{\textbf{tyrannical} adj. using power over people in a way that is cruel and unfair} and selfish a little pig as ever lived. The young English governess who came to teach her to read and write disliked her so much that she gave up her place in three months, and when other governess came to try to fill it they always went away in a shorter time than the first one. So if Mary had not chosen to really want to know how to read books she would never have learned her letters at all.

One frightfully hot morning, when she was about nine years old, she awakened feeling very cross, and she became crosser still when she saw that the servant who stood by her bedside was not her Ayah.

``Why did you come?'' she said to the strange woman. ``I will not let you stay. Send my Ayah to me.''

The woman looked frightened, but she only stammered\footnote{\textbf{stammer} v. to speak with many pauses and repetitions because you hava a speech problem or because you are very nervous, firghtened, etc.} that the Ayah could not come and when Mary threw herself into a passion and beat and kicked her, she looked only more frightened and repeated that it was not possible for the Ayah to come to Missie Sahib.

There was something mysterious in the air that morning. Nothing was done in its regular order and several of the native servants seemed missing, while those whom Mary saw slunk\footnote{\textbf{slink} v. (\textbf{pt./pp.} slunk) to move in a way that does not attract attention especially because you are embarrassed, afraid, or doing something wrong} or hurried about with ashy and scared faces. But no one would tell her anything and her Ayah did not come. She was actually left alone as the morning went on, and at last she wandered out into the garden and began to play by herself under a tree near the veranda\footnote{\textbf{veranda} n. a long, open structure on the outside of a building that has a roof}. She pretended that she was making a flower-bed, and she stuck big scarlet hibiscus\footnote{\textbf{hibiscus} n. a type of shrub that has large colorful flowers} blossoms\footnote{\textbf{blossom} n. a flower especially of a fruit tree} into little heaps of earth, all the time growing more and more angry and muttering\footnote{\textbf{mutter} v. to complain in a quiet or indirect way} to herself the things she would say and the names she would call Saidie when she returned.

``Pig! Pig! Daughter of Pigs!'' she said, because to a native a pig is the worst insult\footnote{\textbf{insult} n. a rude or offensive act or statement} of all.

She was grinding\footnote{\textbf{grind} v. to cause (things) to rub against each other in a forceful way that produces a harsh noise} her teeth and saying this over and over again when she heard her mother come out on the veranda with some one. She was with a fair young man and they stood talking together in low strange voices. Mary knew the fair young man who looked like a boy. She had heard that he was a very young officer who had just come from England. The child stared at him, but she stared most at her mother. She always did this when she had a chance to see her, because the Mem Sahib--Mary used to call her that oftener than anything else--was such a tall, slim\footnote{\textbf{slim} adj. thin in an attractive way}, pretty person and wore such lovely clothes. Her hair was like curly silk and she had a delicate\footnote{\textbf{delicate} adj. attractive because of being soft, gentle, light, etc.} little nose which seemed to be disdaining\footnote{\textbf{disdain} v. to strongly dislike or disapprove of (someone or something)} things, and she had large laughing eyes. All her clothes were thin and floating, and Mary said they were ``full of lace''. They looked fuller of lace than ever this morning, but her eyes were not laughing at all. They were large and scared and lifted imploringly\footnote{\textbf{implore} v. (\textbf{adj.}  imploring, \textbf{adv.} imploringly) to make a very serious or emotional request to (someone)} to the fair boy officer's face.

``Is it so very bad? Oh, is it?'' Mary heard her say.

``Awfully,'' the young man answered in a trembling\footnote{\textbf{tremble} v. to shake slightly because you are afraid, nervous, excited, etc.} voice. ``Awfully, Mrs. Lennox. You ought to have gone to the hills two weeks ago.''

The Mem Sahib wrung\footnote{\textbf{wring} v. (\textbf{pt./pp.} wrung) to get (something) out of someone or something with a lot of effort} her hands.

``Oh, I know I ought!'' she cried. ``I only stayed to go to that silly dinner party. What a fool I was!''

At that very moment such a loud sound of wailing\footnote{\textbf{wail} v. to make a loud, long cry of sadness or pain} broke out from the servants' quarters that she clutched\footnote{\textbf{clutch} v. to hold onto (someone or something) tightly with your hand} the young man's arm, and Mary stood shivering\footnote{\textbf{shiver} v. to shake slightly because you are cold, afraid, etc.} from head to foot. The wailing grew wilder and wilder.

``What is it? What is it?'' Mrs. Lennox gasped\footnote{\textbf{gasp} v. to breathe in suddenly and loudly with your mouth open because of surprise, shock, or pain}.

``Some one has died,'' answered the boy officer. ``You did not say it had broken out among your servants.''

``I did not know!'' the Mem Sahib cried. ``Come with me! Come with me!'' and she turned and ran into the house.

After that, appalling\footnote{\textbf{appall} v. to cause (someone) to feel fear, shock, or disgust} things happened, and the mysteriousness of the morning was explained to Mary. The cholera\footnote{\textbf{cholera} n. a serious disease that causes severe vomiting and diarrhea and that often results in death} had broken out in its most fatal form and people were dying like flies. The Ayah had been taken ill in the night, and it was because she had just died that the servants had wailed in the huts\footnote{\textbf{hut} n. a small and simple house or building}. Before the next day three other servants were dead and others had run away in terror. There was panic on every side, and dying people in all the bungalows\footnote{\textbf{bungalow} n. a house that is all on one level}.

During the confusion and bewilderment\footnote{\textbf{bewilder} v. (\textbf{n.} bewilderment) to confuse (someone) very much} of the second day Mary hid herself in the nursery\footnote{\textbf{nursery} n. a room where children sleep, play, and sometimes taught} and was forgotten by everyone. Nobody thought of her, nobody wanted her, and strange things happened of which she knew nothing. Mary alternately cried and slept through the hours. She only knew that people were ill and that she heard mysterious and frightening sounds. Once she crept\footnote{\textbf{creep} v. (\textbf{pt./pp.} crept) to move slowly with the body close to the ground} into the dining-room and found it empty, though a partly finished meal was on the table and chairs and plates looked as if they had been hastily pushed back when the diners rose suddenly for some reason. The child ate some fruit and biscuits, and being thirsty she drank a glass of wine which stood nearly filled. It was sweet, and she did not know how strong it was. Very soon it made her intensely\footnote{\textbf{intense} adj. (\textbf{adv.} intensely) very great in degree} drowsy\footnote{\textbf{drowsy} adj. tired and ready to fall asleep}, and she went back to her nursery and shut herself in again, frightened by cries she heard in the huts and by the hurrying sound of feet. The wine made her so sleepy that she could scarcely\footnote{\textbf{scarcely} adv. almost not at all} keep her eyes open and she lay down on her bed and knew nothing more for a long time.

Many things happened during the hours in which she slept so heavily, but she was not disturbed by the wails and the sound of things being carried in and out of the bungalow.

When she awakened she lay and stared at the wall. The house was perfectly still. She had never known it to be so silent before. She heard neither voices nor footsteps, and wondered if everybody had got well of the cholera and all the trouble was over. She wondered also who would take care of her now her Ayah was dead. There would be a new Ayah, and perhaps she would know some new stories. Mary had been rather tired of the old ones. She did not cry because her nurse had died. She was not an affectionate\footnote{\textbf{affectionate} adj. feeling or showing love and affection} child and had never cared much for anyone. The noise and hurrying about and wailing over the cholera had frightened her, and she had been angry because no one seemed to remember that she was alive. Everyone was too panic-stricken to think of a little girl no one was fond\footnote{\textbf{fond} adj. feeling or showing love of friendship} of. When people had the cholera it seemed that they remembered nothing but shemselves. But if everyone had got well again, surely some one would remember and come to look for her.
