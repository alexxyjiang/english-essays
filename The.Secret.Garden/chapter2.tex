\chapter{\textit{Mistress Mary Quite Contrary}}
\lettrine{M}{ary} had liked to look at her mother from a distance and she had thought her very pretty, but as she knew very little of her she could scarcely have been expected to love her or to miss her very much when she was gone. She did not miss her at all, in fact, and as she was a self-absorbed\footnote{\textbf{self-absorbed} adj. only caring about and interested in yourself} child she gave her entire thought to herself, as she had always done. If she had been older she would no doubt have been very anxious\footnote{\textbf{anxious} adj. afraid or nervous especially about what may happen} at being left alone in the world, bu she was very young, and as she had always been taken care of, she supposed she always would be. What she thought was that she would like to know if she was going to nice people, who would be polite to her and give her her own way as her Ayah and the other native servants had done.

She knew that she was not going to stay at the English clergyman's\footnote{\textbf{clergyman} n. a man who is a member of the clergy especially in a Christian church} house where she was taken at first. She did not want to stay. The English clergyman was poor and he had five children nearly all the same age and they wore shabby\footnote{\textbf{shabby} adj. in poor condition especially because of age or use} clothes and were always quarreling\footnote{\textbf{quarrel} v. to argue about or disagree with something} and snatching\footnote{\textbf{snatch} v. to take (something) quickly or eagerly} toys from each other. Mary hated their untidy bungalow and was disagreeable to them that after the first day or two nobody would play with her. By the second day they had given her a nickname which made her furious\footnote{\textbf{furious} adj. very angry}.

It was Basil who thought of it first. Basil was a little boy with impudent\footnote{\textbf{impudent} adj. failing to show proper respect and courtesy} blue eyes and a turned-up nose, and Mary hated him. She was playing by herself under a tree, just as she had been playing the day the cholera broke out. She was making heaps of earth and paths for a garden and Basil came and stood near to watch her. Presently he got rather interested and suddenly made a suggestion.

``Why don't you put a heap of stones there and pretend it is a rockery\footnote{\textbf{rockery} n. \textit{(Brit)} rock garden}?'' he said. ``There in the middle,'' and he leaned\footnote{\textbf{lean} v. to bend or move from a straight position} over her to point.

``Go away!'' cried Mary. ``I don't want boys. Go away!''

For a moment Basil looked angry, and then he began to tease\footnote{\textbf{tease} v. to laugh at and criticize (someone) in a way that is either friendly and playful or cruel and unkind}. He was always teasing his sisters. He danced round and round her and made faces and sang and laughed.

\begin{center}
\textit{
``Mistress Mary, quite contrary\footnote{\textbf{contrary} adj. unwilling to obey or behave well},\\
How does your garden grow?\\
With silver bells, and cockle\footnote{\textbf{cockle} n. a type of shellfish with a shell that has two parts and is shaped like a heart } shells,\\
And marigolds\footnote{\textbf{marigold} n. a plant that is grown for its bright yellow or orange flowers} all in a row.''
}
\end{center}

He sang it until the other children heard and laughed, too; and the crosser Mary got, the more they sang `` Mistress Mary, quite contrary''; and after that as long as she stayed with them they called her ``Mistress Mary Quite Contrary'' when they spoke of her to each other, and often when they spoke to her.

``You are going to be sent home,'' Basil said to her, ``at the end of the week. And we're glad of it.''

``I am glad of it, too,'' answered Mary. ``Where is home?''

``She doesn't know where home is!'' said Basil, with seven-year-old scorn\footnote{\textbf{scorn} n. a feeling that someone or somegthing is not worthy of any respect or approval}. ``It's England, of course. Our grandmama lives there and our sister Mabel was sent to her last year. You are not going to your grandmama. You have none. You are going to your uncle. His name is Mr. Archibald Craven.''

``I don't know anything about him,'' snapped\footnote{\textbf{snap} v. to speak using short, angry sentences or phrases} Mary.

``I know you don't,'' Basil answered. ``You don't know anything. Girls never do. I heard father and mother talking about him. He lives in a great, big, desolate\footnote{\textbf{desolate} adj. lacking the people, plants, animals, etc., that make people feel welcome in a place} old house in the country and no one goes near him. He's so cross he won't let them, and they wouldn't come if he would let them. He's hunchback\footnote{\textbf{hunchback} n. a back in which the spine is curved in an abnormal way}, and he's horrid\footnote{\textbf{horrid} adj. very unpleasant}.''

``I don't believe you,'' said Mary; and she turned her back and stuck her fingers in her ears, because she would not listen any more.

But she thought over it a great deal afterward; and when Mrs. Crawford told her that night that she was going to sail away to England in a few days and go to her uncle, Mr. Archibald Craven, who lived at Misselthwaite Manor, she looked so stony and stubbornly\footnote{\textbf{stubborn} adj. (\textbf{adv.} stubbornly) refusing to change your ideas or to stop doing something} uninterested that they did not know what to think about her. They tried to be kind to her, but she only turned her face away when Mrs. Crawford attempted to kiss her, and held herself stiffly when Mr. Crawford patted her shoulder.

``She is such a plain child,'' Mrs. Crawford said pityingly, afterward. ``And her mother was such a pretty creature. She had a very pretty manner, too, and Mary has the most unattractive ways I ever saw in a child. The children call her `Mistress Mary Ouite Contrary', and though it's naughty\footnote{\textbf{naughty} adj. behaving badly--used especially to describe a child who does not behave properly or obey a parent, teacher, etc.} of them, one can't help understanding it.''

``Perhaps if her mother had carried her pretty face and her pretty manners oftener into the nursery Mary might have learned some pretty ways too. It is very sad, now the poor beautiful thing is gone, to remember that many people never knew that she had a child at all.''

``I believe she scarcely ever looked at her,'' sighed Mrs. Crawford. ``When Ayah was dead there was no one to give a thought to the little thing. Think of the servants running away and leaving her all alone in that deserted bungalow. Colonel McGrew said he nearly jumped out of his skin when he opened the door and found her standing by herself in the middle of the room.''

Mary made the long voyage to England under the care of an officer's wife, who was taking her children to leave them in a boarding-school. She was very much absorbed in her own little boy and girl, and was rather glad to hand the child over to the woman Mr. Archibald Craven sent to meet her, in London. The woman was his housekeeper at Misselthwaite Manor, and her name was Mrs. Medlock. She was a stout\footnote{\textbf{stout} adj. having a large body that is wide with fat or muscles} woman, with very red cheeks and sharp black eyes. She wore a very purple bonnet\footnote{\textbf{bonnet} n. a hat that ties under the chin} with purple velvet\footnote{\textbf{velvet} n. a soft type of cloth that has short raised fibers on one side} flowers which stuck up and trembled when she moved her head. Mary did not like her at all, but as she very seldom\footnote{\textbf{seldom} adv. almost never} liked people there was nothing remarkable in that; besides which it was very evident Mrs. Medlock did not think much of her.

``My word! she's a plain little piece of goods!'' she said. ``And we'd heard that her mother was a beauty. She hasn't handed much of it down, has she, ma'am?''

``Perhaps she will improve as she grows older,'' the officer's wife said good-naturedly. ``If she were not so sallow\footnote{\textbf{sallow} adj. slightly yellow in a way that does not look healthy} and had a nicer expression... her features are rather good. Children alter so much.''
