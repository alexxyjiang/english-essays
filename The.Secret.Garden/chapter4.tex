\chapter{\textit{Martha}}
\lettrine{W}{hen} she opened her eyes in the morning it was because a young house-maid had come into her room to light the fire and was kneeling\footnote{\textbf{kneel} v. (\textbf{pt./pp.} knelt) to move your body so that one or both of your knees are on the floor} on the hearth-rug\footnote{\textbf{hearth} n. the floor in front of or inside a fireplace} raking\footnote{\textbf{rake} n. a tool that has a series of metal, wooden, or plastic pieces at the end of a long handle and that is used to gather leaves, break apart soil, make ground smooth, etc. \textbf{rake} v. to use a rake to gather leaves, break apart soil, make ground smooth, etc.} out the cinders\footnote{\textbf{cinder} n. a very small piece of burned material (such as wood or coal)} noisily. Mary lay and watched her for a few moments and then began to look about the room. She had never seen a room at all like it and thought it curious and gloomy. The walls were covered with tapestry\footnote{\textbf{tapestry} n. a heavy cloth that has designs or pictures woven into it and that is used for wall hangings, curtains, etc.} with a forest scene embroidered\footnote{\textbf{embroider} v. to sew a design on a piese of cloth} on it. There were fantastically\footnote{\textbf{fantastic} adj. (\textbf{adv.} fantastically) extremely good} dressed people under the trees and in the distance there was a glimpse of the turrets\footnote{\textbf{turret} n. a small tower on a building} of a castle. There were hunters and horses and dogs and ladies. Mary felt as if she were in the forest with them. Ouot of a deep window she could see a great climbing stretch of land which seemed to have no trees on it, and to look rather like an endless, dull, purplish sea.

``What is that?'' she said, pointing out of the window.

Martha, the young house-maid, who had just risen to her feet, looked and pointed also.

``That there?'' she said.

``Yes.''

``That's th' moor,'' with a good-natured grin\footnote{\textbf{grin} v. (\textbf{n.} grin) to smile widely}. ``Dose tha' like it?''

``No,'' answered Mary. ``I hate it.''

``That's because tha'rt not used to it,'' Martha said, going back to her hearth. ``Tha' thinks it's too big an' bare\footnote{\textbf{bare} adj. not having a covering} now. But tha' will like it.''

``Do you?'' inquired\footnote{\textbf{inquire} v. to ask for information} Mary.

``Aye, that I do,'' answered Martha, cheerfully polishing\footnote{\textbf{polish} v. to make (something) smooth and shiny by rubbing it} away at the grate\footnote{\textbf{grate} n. a metal frame with bars across it that is used in a fireplace or to cover an opening}. ``I just love it. It's none bare. It's covered wi' growin' things as smells sweet. It's fair lovely in spring an' summer when th' gorse an' broom an' heather's in flower. It smells o' honey an' there's such a lot o' fresh air--an' th' sky looks so high an' th' bees an' skylarks\footnote{\textbf{skylark} n. a small bird of Europe, Asia, and northern Africa that sings while it flies} makes such a nice noise hummin'\footnote{\textbf{hum} v. to make a low continuous sound} an' singin'. Eh! I wouldn't live away from th' moor for anythin'.''

Mary listened to her with a grave\footnote{\textbf{grave} adj. very serious: requiring or causing serious thought or concern}, puzzled expression. The native servants she had been used to in India were not in the least like this. They were obsequious\footnote{\textbf{obsequious} adj. too eager to help or obey someone important} and servile\footnote{\textbf{servile} adj. very obedient and trying too hard to please someone} and did not presume to talk to their masters as if they were their equals. They made salaams\footnote{\textbf{salaam} n. a Muslim greeting} and called them ``protector of the poor'' and names of that sort. Indian servants were commanded to do things, not asked. It was not the custom to say ``please'' and ``thank you'' and Mary had always slapped her Ayah in the face when she was angry. She wondered a little what this girl would do if one slapped her in the face. She was a round, rosy, good-natured-looking creature, but she had a sturdy\footnote{\textbf{sturdy} adj. having or showing mental or emotional strength} way which made Mistress Mary wonder if she might not even slap back--if the person who slapped her was only a little girl.

``You are a strange servant,'' she said from her pillows, rather haughtily\footnote{\textbf{haughty} adj. (\textbf{adv.} haughtily) having or showing the insulting attitude of people who think that they are better, smarter, or more important than other people}.

Martha sat up on her heels\footnote{\textbf{heel} n. the back part of your foot that is below the ankle}, with her blackingbrush in her hand, and laughed, without seeming the least out of temper.

``Eh! I know that,'' she said. ``If there was a grand Missus at Misselthwaite I should never have been even one of th' under house-maids. I might have been let to be scullery-maid\footnote{\textbf{scullery} n. a room that is near the kitchen in a large and usually old house and that is used for washing dishes, doing messy kitchen tasks, etc.} but I'd never have been let upstairs. I'm too common an' I talk too much Yorkshire. But this is a funny house for all it's so grand. Seems like there's neither Master nor Mistress except Mr. Pitcher an' Mrs. Medlock. Mr. Craven, he won't be troubled about anythin' when he's here, an' he's nearly always away. Mrs. Medlock gave me th' place out o' kindness. She told me she could never have done it if Misselthwaite had been like other big houses.''

``Are you going to be my servant?'' Mary asked, still in her imperious\footnote{\textbf{imperious} adj. having or showing the proud and unpleasant attitude of someone who gives orders and expects other people to obey them} little Indian way.

Martha began to rub her grate again.

``I'm Mrs. Medlock's servant,'' she said stoutly. ``An' she's Mr. Craven's--but I'm to do the house-maid's work up here an' wait on you a bit. But you won't need much waitin' on.''

``Who is going to dress me?'' demanded Mary.

Martha sat up on her heels again and stared. She spoke in broad Yorkshire in her amazement.

``Canna' tha' dress thysel'!'' she said.

``What do you mean? I don't understand your language,'' said Mary.

``Eh! I forgot,'' Martha said. ``Mrs. Medlock told me I'd have to be careful or you wouldn't know what I was sayin'. I mean can't you put on your own clothes?''

``No,'' answered Mary, quite indignantly\footnote{\textbf{indignant} adj. (\textbf{adv.} indignantly) feeling or showing anger because of something that is unfair or wrong}. ``I never did in my life. My Ayah dressed me, of course.''

``Well,'' said Martha, evidently not in the least aware that she was impudent, ``it's time tha' should learn. Tha' cannot begin younger. It'll do thee good to wait on thysel' a bit. My mother always said she couldn't see why grand people's children didn't turn out fair fools--what with nurses an' bein' washed an' dressed an' took out to walk as if they was puppies!''

``It is different in India,'' said Mistress Mary disdainfully. She could scarcely\footnote{\textbf{scarcely} adv. almost not at all} stand this.

But Martha was not at all crushed.

``Eh! I can see it's different,'' she answered almost sympathetically\footnote{\textbf{sympathetic} adj. (\textbf{adv.} sympathetically) feeling or showing concern about someone who is in a bad situation}. ``I dare say it's because there's such a lot o' blacks there instead o' respectable white people. When I heard you was comin' from India I thought you was a black too.''

Mary sat up in bed furious.

``What!'' she said. ``What! You thought I was a native. You--you daughter of a pig!''

Martha stared and looked hot.

``Who are you callin' names?'' she said. ``You needn't be so vexed\footnote{\textbf{vex} v. to annoy or worry (someone)}. That's not th' way for a young lady to talk. I've nothin' against th' blacks. When you read about 'em in tracts they're always very religious\footnote{\textbf{religious} adj. believing in a god or a group of gods and following the rules of a religion}. You always read as a black's a man an' a brother. I've never seen a black an' I was fair pleased to think I was goin' to see one close. When I come in to light your fire this mornin' I crep' up to your bed an' pulled th' cover back careful to look at you. An' there you was,'' disappointedly, ``no more black than me--for all you're so yeller\footnote{\textbf{yell} v. (\textbf{n.} yeller) to say (something) very loudly especially because you are angry, surprised, or are trying to get someone's attention}.''

Mary did not even try to control her rage\footnote{\textbf{rage} n. a strong feeling of anger that is difficult to control} and humiliation\footnote{\textbf{humiliate} v. (\textbf{n.} humiliation) to make (someone) feel very ashamed or foolish}.

``You thought I was a native! You dared! You don't know anything about natives! They are not people--they're servants who must salaam to you. You know nothing about India. You know nothing about anything!''

She was in such a rage and felt so helpless before the girl's simple stare, and somehow she suddenly felt so horribly lonely and far away from everything she understood and which understood her, that she threw herself face downward on the pillows and burst\footnote{\textbf{burst} v. to break open or into pieces in a sudden and violent way} into passionate\footnote{\textbf{passionate} adj. having, showing, or expressing strong emotions or beliefs} sobbing. She sobbed so unrestrainedly\footnote{\textbf{unrestrained} adj. (\textbf{adv.} unrestrainedly) not held in place by a belt, seat, device, etc.} that good-natured Yorkshire Martha was a little firghtened and quite sorry for her. She went to the bed and bent over her.

``Eh! You mustn't cry like that there!'' She begged. ``You mustn't for sure. I didn't know you'd be vexed. I don't know anythin' about anythin'--just like you said. I beg you pardon, Miss. Do stop cryin'.''

There was something comforting\footnote{\textbf{comfort} v. (\textbf{adj.} comforting) to cause (someone) to feel less worried, upset, frightened, etc.} and really friendly in her queer Youkshire speech and sturdy way which had a good effect on Mary. She gradually\footnote{\textbf{gradual} adj. (\textbf{adv.} gradually) moving or changing in small amounts} ceased crying and became quiet. Martha looked relieved\footnote{\textbf{relieved} adj. feeling relaxed and happy because something difficult or unpleasant has been stopped, avoided, or made easier}.

``It's time for thee to get up now,'' she said. ``Mrs. Medlock said I was to carry tha' breakfast an' tea an' dinner into th' room next to this. It's been made into a nursery for thee. I'll help thee on with thy clothes if tha'll get out o' bed. If th' buttons are at th' back tha' cannot button them up tha' self.''

When Mary at last decided to get up, the clothes Martha took from the wardrobe\footnote{\textbf{wardrobe} n. a collection of clothes that a person owns or wears} were not the ones she had worn when she arrived the night before with Mrs. Medlock.

``Those are not mine,'' she said. ``Mine are black.''

She looked the thick white wool\footnote{\textbf{wool} n. the soft, thick hair of sheep and some other animals} coat and dress over, and added with cool approval:

``Those are nicer than mine.''

``These are th' ones tha' must put on,'' Martha answered. ``Mr. Craven ordered Mrs. Medlock to get 'em in London. He said `I won't have a child dressed in black wanderin' about like a lost soul,' he said. `It'd make the place sadder than it is. Put color on her,' Mother she said she knew what he meant. Mother always knows what a body means. She doesn't hold with black hersel'.''

``I hate black things,'' said Mary.

The dressing process was one which taught them both something. Martha had ``buttoned up'' her little sisters and brothers but she had never seen a child who stood still and waited for another person to do things for her as if she had neither hands nor feet of her own.

``Why doesn't tha' put on tha' own shoes?'' she said when Mary quietly held out her foot.

``My Ayah did it,'' answered Mary, staring. ``It was the custom.''

She said that very often--``It was the custom.'' The native servants were always saying it. If one told them to do a thing their ancestors had not done for a thousand years they gazed at one mildly\footnote{\textbf{mild} adj. (\textbf{adv.} mildly) gentle in nature or behavior} and said. ``It is not the custom,'' and one knew that was the end of the matter.

It had not been the custom that Mistress Mary should do anything but stand and allow herself to be dressed like a doll, but before she was ready for breakfast she began to suspect\footnote{\textbf{suspect} v. to think that (something, especially something bad) possibly exists, is true, will happen, etc.)} that her life at Misselthwaite Manor would end by teaching her a number of things quite new to her--things such as putting on her own shoes and stockings, and picking up things she let fall. If Martha had been a well-trained fine young lady's maid she would have been more subservient\footnote{\textbf{subservient} adj. very willing or too willing to obey someone else} and respectful and would have known that it was her business to brush hair, and botton boots, and pick things up and lay them away. She was, however, only an untrained Yorkshire rustic\footnote{\textbf{rustic} adj. of, relating to, or suitable for the country or people who live in the country} who had been brought up in a moorland cottage with a swarm\footnote{\textbf{swarm} n. a very large of insects moving together} of little brothers and sisters who had never dreamed of doing anything but waiting on themselves and on the younger ones who were either babies in arms or just learning to totter\footnote{\textbf{totter} v. to move or walk in a slow and unsteady way} about and tumble\footnote{\textbf{tumble} v. to fall down suddenly and quickly} over things.
