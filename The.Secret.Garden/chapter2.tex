\chapter{\textit{Mistress Mary Quite Contrary}}
\lettrine{M}{ary} had liked to look at her mother from a distance and she had thought her very pretty, but as she knew very little of her she could scarcely have been expected to love her or to miss her very much when she was gone. She did not miss her at all, in fact, and as she was a self-absorbed\footnote{\textbf{self-absorbed} adj. only caring about and interested in yourself} child she gave her entire thought to herself, as she had always done. If she had been older she would no doubt have been very anxious\footnote{\textbf{anxious} adj. afraid or nervous especially about what may happen} at being left alone in the world, bu she was very young, and as she had always been taken care of, she supposed she always would be. What she thought was that she would like to know if she was going to nice people, who would be polite to her and give her her own way as her Ayah and the other native servants had done.

She knew that she was not going to stay at the English clergyman's\footnote{\textbf{clergyman} n. a man who is a member of the clergy especially in a Christian church} house where she was taken at first. She did not want to stay. The English clergyman was poor and he had five children nearyly all the same age and they wore shabby\footnote{\textbf{shabby} adj. in poor condition especially because of age or use} clothes and were always quarreling\footnote{\textbf{quarrel} v. to argue about or disagree with something} and snatching\footnote{\textbf{snatch} v. to take (something) quickly or eagerly} toys from each other. Mary hated their untidy bungalow and was disagreeable to them that after the first day or two nobody would play with her. By the second day they had given her a nickname which made her furious\footnote{\textbf{furious} adj. very angry}.
