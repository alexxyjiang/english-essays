\chapter{\textit{Mistress Mary Quite Contrary}}
\lettrine{M}{ary} had liked to look at her mother from a distance and she had thought her very pretty, but as she knew very little of her she could scarcely have been expected to love her or to miss her very much when she was gone. She did not miss her at all, in fact, and as she was a self-absorbed\footnote{\textbf{self-absorbed} adj. only caring about and interested in yourself} child she gave her entire thought to herself, as she had always done. If she had been older she would no doubt have been very anxious\footnote{\textbf{anxious} adj. afraid or nervous especially about what may happen} at being left alone in the world, bu she was very young, and as she had always been taken care of, she supposed she always would be. What she thought was that she would like to know if she was going to nice people, who would be polite to her and give her her own way as her Ayah and the other native servants had done.

She knew that she was not going to stay at the English clergyman's\footnote{\textbf{clergyman} n. a man who is a member of the clergy especially in a Christian church} house where she was taken at first. She did not want to stay. The English clergyman was poor and he had five children nearly all the same age and they wore shabby\footnote{\textbf{shabby} adj. in poor condition especially because of age or use} clothes and were always quarreling\footnote{\textbf{quarrel} v. to argue about or disagree with something} and snatching\footnote{\textbf{snatch} v. to take (something) quickly or eagerly} toys from each other. Mary hated their untidy bungalow and was disagreeable to them that after the first day or two nobody would play with her. By the second day they had given her a nickname which made her furious\footnote{\textbf{furious} adj. very angry}.

It was Basil who thought of it first. Basil was a little boy with impudent\footnote{\textbf{impudent} adj. failing to show proper respect and courtesy} blue eyes and a turned-up nose, and Mary hated him. She was playing by herself under a tree, just as she had been playing the day the cholera broke out. She was making heaps of earth and paths for a garden and Basil came and stood near to watch her. Presently he got rather interested and suddenly made a suggestion.

``Why don't you put a heap of stones there and pretend it is a rockery\footnote{\textbf{rockery} n. \textit{(Brit)} rock garden}?'' he said. ``There in the middle,'' and he leaned\footnote{\textbf{lean} v. to bend or move from a straight position} over her to point.

``Go away!'' cried Mary. ``I don't want boys. Go away!''

For a moment Basil looked angry, and then he began to tease\footnote{\textbf{tease} v. to laugh at and criticize (someone) in a way that is either friendly and playful or cruel and unkind}. He was always teasing his sisters. He danced round and round her and made faces and sang and laughed.

\begin{center}
\textit{
``Mistress Mary, quite contrary\footnote{\textbf{contrary} adj. unwilling to obey or behave well},\\
How does your garden grow?\\
With silver bells, and cockle\footnote{\textbf{cockle} n. a type of shellfish with a shell that has two parts and is shaped like a heart } shells,\\
And marigolds\footnote{\textbf{marigold} n. a plant that is grown for its bright yellow or orange flowers} all in a row.''
}
\end{center}

He sang it until the other children heard and laughed, too; and the crosser Mary got, the more they sang `` Mistress Mary, quite contrary''; and after that as long as she stayed with them they called her ``Mistress Mary Quite Contrary'' when they spoke of her to each other, and often when they spoke to her.

``You are going to be sent home,'' Basil said to her, ``at the end of the week. And we're glad of it.''

``I am glad of it, too,'' answered Mary. ``Where is home?''

``She doesn't know where home is!'' said Basil, with seven-year-old scorn\footnote{\textbf{scorn} n. a feeling that someone or somegthing is not worthy of any respect or approval}. ``It's England, of course. Our grandmama lives there and our sister Mabel was sent to her last year. You are not going to your grandmama. You have none. You are going to your uncle. His name is Mr. Archibald Craven.''

``I don't know anything about him,'' snapped\footnote{\textbf{snap} v. to speak using short, angry sentences or phrases} Mary.

``I know you don't,'' Basil answered. ``You don't know anything. Girls never do. I heard father and mother talking about him. He lives in a great, big, desolate\footnote{\textbf{desolate} adj. lacking the people, plants, animals, etc., that make people feel welcome in a place} old house in the country and no one goes near him. He's so cross he won't let them, and they wouldn't come if he would let them. He's hunchback\footnote{\textbf{hunchback} n. a back in which the spine is curved in an abnormal way}, and he's horrid\footnote{\textbf{horrid} adj. very unpleasant}.''

``I don't believe you,'' said Mary; and she turned her back and stuck her fingers in her ears, because she would not listen any more.

But she thought over it a great deal afterward; and when Mrs. Crawford told her that night that she was going to sail away to England in a few days and go to her uncle, Mr. Archibald Craven, who lived at Misselthwaite Manor, she looked so stony and stubbornly\footnote{\textbf{stubborn} adj. (\textbf{adv.} stubbornly) refusing to change your ideas or to stop doing something} uninterested that they did not know what to think about her. They tried to be kind to her, but she only turned her face away when Mrs. Crawford attempted to kiss her, and held herself stiffly when Mr. Crawford patted her shoulder.

``She is such a plain child,'' Mrs. Crawford said pityingly, afterward. ``And her mother was such a pretty creature. She had a very pretty manner, too, and Mary has the most unattractive ways I ever saw in a child. The children call her `Mistress Mary Ouite Contrary', and though it's naughty\footnote{\textbf{naughty} adj. behaving badly--used especially to describe a child who does not behave properly or obey a parent, teacher, etc.} of them, one can't help understanding it.''

``Perhaps if her mother had carried her pretty face and her pretty manners oftener into the nursery Mary might have learned some pretty ways too. It is very sad, now the poor beautiful thing is gone, to remember that many people never knew that she had a child at all.''

``I believe she scarcely ever looked at her,'' sighed Mrs. Crawford. ``When Ayah was dead there was no one to give a thought to the little thing. Think of the servants running away and leaving her all alone in that deserted bungalow. Colonel McGrew said he nearly jumped out of his skin when he opened the door and found her standing by herself in the middle of the room.''

Mary made the long voyage to England under the care of an officer's wife, who was taking her children to leave them in a boarding-school. She was very much absorbed in her own little boy and girl, and was rather glad to hand the child over to the woman Mr. Archibald Craven sent to meet her, in London. The woman was his housekeeper at Misselthwaite Manor, and her name was Mrs. Medlock. She was a stout\footnote{\textbf{stout} adj. having a large body that is wide with fat or muscles} woman, with very red cheeks and sharp black eyes. She wore a very purple bonnet\footnote{\textbf{bonnet} n. a hat that ties under the chin} with purple velvet\footnote{\textbf{velvet} n. a soft type of cloth that has short raised fibers on one side} flowers which stuck up and trembled when she moved her head. Mary did not like her at all, but as she very seldom\footnote{\textbf{seldom} adv. almost never} liked people there was nothing remarkable in that; besides which it was very evident Mrs. Medlock did not think much of her.

``My word! she's a plain little piece of goods!'' she said. ``And we'd heard that her mother was a beauty. She hasn't handed much of it down, has she, ma'am?''

``Perhaps she will improve as she grows older,'' the officer's wife said good-naturedly. ``If she were not so sallow\footnote{\textbf{sallow} adj. slightly yellow in a way that does not look healthy} and had a nicer expression... her features are rather good. Children alter so much.''

``She'll have to alter a good deal,'' answered Mrs. Medlock. ``And there's nothing likely to improve children at Misselthwaite--if you ask me!''

They thought Mary was not listening because she was standing a little apart from them at the window of the private hotel they had gone to. She was watching the passing buses and cabs and people, but she heard quite well and was made very curious about her uncle and the place he lived in. What sort of a place was it, and what would he be like? What was a hunchback? She had never seen one. Perhaps there were none in India.

Since she had been living in other people's houses and had had no Ayah, she had begun to feel lonely and to think queer thoughts which were new to her. She had begun to wonder why she had never seemed to belong to anyone even when her father and mother had been alive. Other children seemed to belong to their fathers and mothers, but she had never seemed to really be anyone's little girl. She had had servants, and food and clothes, but no one had taken any notice of her. She did not know that this was because she was a disagreeable child; but then, of course, she did not know she was disagreeable. She often thought that other people were, but she did not know that she was so herself.

She thought Mrs. Medlock the most disagreeable person she had ever seen, with her common, highly colored face and her common fine bonnet. When the next day they set out on their journey to Yorkshire, she walked through the station to the railway carriage with her head up and trying to keep as far away from her as she could, because she did not want to seem to belong to her. It would have made her angry to think people imagined she was her little girl.

But Mrs. Medlock was not in the least disturbed by her and her thoughts. She was the kind of woman who would ``stand no nonsense from young ones''. At least, that is what she would have said if she had been asked. She had not wanted to go to London just when her sister Maria's daughter was going to be married, but she had a comfortable, well paid place as housekeeper at Misselthwaite Manor and the only way in which she coule keep it was to do at once what Mr. Archibald Craven told her to do. She never dared even to ask a question.

``Captain Lennox and his wife died of the cholera,'' Mr. Craven had said in his short, cold way. ``Captain Lennox was my wife's brother and I am their daughter's guardian\footnote{\textbf{guardian} n. someone who takes care of another person or of another person's property}. The child is to be brought here. You must go to London and bring her yourself.''

So she packed her small trunk and made the journey.

Mary sat in her corner of the railway carriage and looked plain and fretful. She had nothing to read or look at, and she had folded her thin little black-gloved hands in her lap. Her black dress made her look yellower than ever, and her limp\footnote{\textbf{limp} adj. having an unpleasantly soft or weak quality} light hair straggled\footnote{\textbf{straggle} v. to move away or spread out from others in a disorganized way} from under her black cr\^{e}pe\footnote{\textbf{cr\^{e}pe} n. a thin often silk or cotton cloth that has many very small wrinkles all over its surface} hat.

``A more marred-looking\footnote{\textbf{mar} v. (\textbf{pt./pp.} marred) to ruin the beauty or perfection of (something)} young one I never saw in my life,'' Mrs. Medlock thought. (Marred is a Yorkshire word that means spoiled and pettish.) She had never seen a child who sat so still without doing anything; and at last she got tired of watching her and began to talk in a brisk\footnote{\textbf{brisk} adj. moving or speaking quickly }, hard voice.

``I suppose I may as well tell you something about where you are going to,'' she said. ``Do you know anything about your uncle?''

``No,'' said Mary.

``Never heard your father and mother talk about him?''

``No,'' said Mary, frowning. She frowned because she remembered that her father and mother had never talked to her about anything in particular. Certainly they had never told her things.

``Humph,'' muttered Mrs. Medlock, staring at her queer, unresponsive little face. She did not say any more for a few moments and then she began again.

``I suppose you might as well be told something--to prepare you. You are going to a queer place.''

Mary said nothing at all, and Mrs. Medlock looked rather discomfited\footnote{\textbf{discomfit} v. to make (someone) confused or upset} by her apparent indifference, but, after taking a breath, she went on.

``Not but that it's a grand big place in a gloomy\footnote{\textbf{gloomy} adj. something dark} way, and Mr. Craven's proud of it in his way--and that's gloomy enough, too. The house is six hundred years old and it's on the edge of the moor\footnote{\textbf{moor} n. a broad area of open land that is not good for farming}, and there's near a hundred rooms in it, though most of them's shut up and locked. And there's pictures and fine old furniture and things that's been there for ages, and there's a big park round it and gardens and trees with branches trailing to the ground--some of them.'' She paused and took another breath. ``But there's nothing else,'' she ended suddenly.

Mary had begun to listen in spite\footnote{\textbf{spite} n. a desire to harm, anger, or defeat another person especially because you feel that you have been treated wrongly in some way} of herself. It all sounded so unlike India, and anything new rather attracted her. But she did not intend to look as if she were interested. That was one of her unhappy, disagreeable ways. So she sat still.

``Well,'' said Mrs. Medlock. ``What do you think of it?''

``Nothing,'' she answered. ``I know nothing about such places.''

That made Mrs. Medlock laugh a short sort of laugh.

``Eh!'' she said, ``but you are like an old woman. Don't you care?''

``It doesn't matter,'' said Mary, ``whether I care or not.''

``You are right enough there,'' said Mrs. Medlock. ``It doesn't. What you're to be kept at Misselthwaite Manor for I don't know, unless because it's the easiest way. \textit{He's} not going to trouble himself about you, that's sure and certain. He never troubles himself about no one.''

She stopped herself as if she had just remembered something in time.

``He's got a crooked\footnote{\textbf{crooked} adj. not straight} back,'' she said. ``That set him wrong. He was a sour young man and got no good of all his money and big place till he was married.''

Mary's eyes turned toward her in spite of her intention not to seem to care. She had never thought of the hunchback's being married and she was a trifle\footnote{\textbf{trifle} n. something that does not have much value or importance} surprised. Mrs. Medlock saw this, and as she was a talkative woman she continued with more interest. This was one way of passing some of the time, at any rate.

``She was a sweet, pretty thing and he'd have walked the world over to get her a blade o' grass she wanted. Nobody thought she'd marry him, but she did, and people said she married him for his money. But she didn't--she didn't,'' positively. ``When she died--''

Mary gave a little involuntary\footnote{\textbf{involuntary} adj. not voluntary: such as not done by choice} jump.

``Oh! did she die!'' she exclaimed, quite without meaning to. She had just remembered a French fairy\footnote{\textbf{fairy} n. a creature that looks like a very small human being, has magic powers, and sometimes has wings} story she had once read called ``Riquet \`{a} la Houppe'' (Riquet with the Tuft\footnote{\textbf{tuft} n. a small bunch of feathers, hairs, grass, etc., that grow close together}). It had been about a poor hunchback and a beautiful princess and it had made her suddenly sorry for Mr. Archibald Craven.

``Yes, she died,'' Mrs. Medlock answered. ``And it made him queerer than ever. He cares about nobody. He won't see people. Most of the time he goes away, and when he is at Misselthwaite he shuts himself up in the West Wing and won't let any one but Pitcher see him. Pitcher's an old fellow, but he took care of him when he was a child and he knows his ways.''

It sounded like something in a book and it did not make Mary feel cheerful. A house with a hundred rooms, nearly all shut up and with their doors locked--a house on the edge of a moor--whatsoever\footnote{\textbf{whatsoever} adj. of any kind or amount at all} a moor was--sounded dreary\footnote{\textbf{dreary} adj. causing unhappiness or sad feelings}. A man with a crooked back who shut himself up also! She stared out of the window with her lips pinched\footnote{\textbf{pinch} v. to squeeze or press (something) together with your thumb and finger} together, and it seemed quite natural that the rain should have begun to pour\footnote{\textbf{pour} v. to rain heavily} down in gray slanting\footnote{\textbf{slant} v. to not be level or straight up and down} lines and splash\footnote{\textbf{splash} v. to cause (water or another liquid) to move in a noisy way or messy way} and stream down the window-panes. If the pretty wife had been alive she might have made things cheerful by being something like her own mother and by running in and out and going to parties as she had done in frocks ``full of lace''. But she was not there any more.

``You needn't expect to see him, because ten to one you won't,'' said Mrs. Medlock. ``And you mustn't expect that there will be people to talk to you. You'll have to play about and look after yourself. You'll be told what rooms you can go into and what rooms you're to keep out of. There's gardens enough. But when you're in the house don't go wandering and poking\footnote{\textbf{poke} v. to push your finger or something thin or pointed into or at someone or something} about. Mr. Craven won't have it.''

``I shall not want to go poking about,'' said sour little Mary; and just as suddenly as she had begun to be rather sorry for Mr. Archibald Craven she began to cease\footnote{\textbf{cease} v. to stop happening} to be sorry and to think he was unpleasant enough to deserve all that had happened to him.

And she turned her face toward the streaming panes of the window of the railway carriage and gazed\footnote{\textbf{gaze} v. to look at someone or something in a steady way and usually for a long time} out at the gray rainstorm which looked as if it would go on forever and ever. She watched it so long and steadily that the grayness grew heavier and heavier before her eyes and she fell asleep.
