\chapter{\textit{Martha}}
\lettrine{W}{hen} she opened her eyes in the morning it was because a young house-maid had come into her room to light the fire and was kneeling\footnote{\textbf{kneel} v. (\textbf{pt./pp.} knelt) to move your body so that one or both of your knees are on the floor} on the hearth-rug\footnote{\textbf{hearth} n. the floor in front of or inside a fireplace} raking\footnote{\textbf{rake} n. a tool that has a series of metal, wooden, or plastic pieces at the end of a long handle and that is used to gather leaves, break apart soil, make ground smooth, etc. \textbf{rake} v. to use a rake to gather leaves, break apart soil, make ground smooth, etc.} out the cinders\footnote{\textbf{cinder} n. a very small piece of burned material (such as wood or coal)} noisily. Mary lay and watched her for a few moments and then began to look about the room. She had never seen a room at all like it and thought it curious and gloomy. The walls were covered with tapestry\footnote{\textbf{tapestry} n. a heavy cloth that has designs or pictures woven into it and that is used for wall hangings, curtains, etc.} with a forest scene embroidered\footnote{\textbf{embroider} v. to sew a design on a piese of cloth} on it. There were fantastically\footnote{\textbf{fantastic} adj. (\textbf{adv.} fantastically) extremely good} dressed people under the trees and in the distance there was a glimpse of the turrets\footnote{\textbf{turret} n. a small tower on a building} of a castle. There were hunters and horses and dogs and ladies. Mary felt as if she were in the forest with them. Ouot of a deep window she could see a great climbing stretch of land which seemed to have no trees on it, and to look rather like an endless, dull, purplish sea.

``What is that?'' she said, pointing out of the window.

Martha, the young house-maid, who had just risen to her feet, looked and pointed also.

``That there?'' she said.

``Yes.''

``That's th' moor,'' with a good-natured grin\footnote{\textbf{grin} v. (\textbf{n.} grin) to smile widely}. ``Dose tha' like it?''

``No,'' answered Mary. ``I hate it.''

``That's because tha'rt not used to it,'' Martha said, going back to her hearth. ``Tha' thinks it's too big an' bare\footnote{\textbf{bare} adj. not having a covering} now. But tha' will like it.''

``Do you?'' inquired\footnote{\textbf{inquire} v. to ask for information} Mary.

``Aye, that I do,'' answered Martha, cheerfully polishing\footnote{\textbf{polish} v. to make (something) smooth and shiny by rubbing it} away at the grate\footnote{\textbf{grate} n. a metal frame with bars across it that is used in a fireplace or to cover an opening}. ``I just love it. It's none bare. It's covered wi' growin' things as smells sweet. It's fair lovely in spring an' summer when th' gorse an' broom an' heather's in flower. It smells o' honey an' there's such a lot o' fresh air--an' th' sky looks so high an' th' bees an' skylarks\footnote{\textbf{skylark} n. a small bird of Europe, Asia, and northern Africa that sings while it flies} makes such a nice noise hummin'\footnote{\textbf{hum} v. to make a low continuous sound} an' singin'. Eh! I wouldn't live away from th' moor for anythin'.''

Mary listened to her with a grave\footnote{\textbf{grave} adj. very serious: requiring or causing serious thought or concern}, puzzled expression. The native servants she had been used to in India were not in the least like this. They were obsequious\footnote{\textbf{obsequious} adj. too eager to help or obey someone important} and servile\footnote{\textbf{servile} adj. very obedient and trying too hard to please someone} and did not presume to talk to their masters as if they were their equals. They made salaams\footnote{\textbf{salaam} n. A low bow as a ceremonial act of deference.} and called them ``protector of the poor'' and names of that sort. Indian servants were commanded to do things, not asked. It was not the custom to say ``please'' and ``thank you'' and Mary had always slapped her Ayah in the face when she was angry. She wondered a little what this girl would do if one slapped her in the face. She was a round, rosy, good-natured-looking creature, but she had a sturdy\footnote{\textbf{sturdy} adj. having or showing mental or emotional strength} way which made Mistress Mary wonder if she might not even slap back--if the person who slapped her was only a little girl.

``You are a strange servant,'' she said from her pillows, rather haughtily\footnote{\textbf{haughty} adj. (\textbf{adv.} haughtily) having or showing the insulting attitude of people who think that they are better, smarter, or more important than other people}.

Martha sat up on her heels\footnote{\textbf{heel} n. the back part of your foot that is below the ankle}, with her blackingbrush in her hand, and laughed, without seeming the least out of temper.

``Eh! I know that,'' she said. ``If there was a grand Missus at Misselthwaite I should never have been even one of th' under house-maids. I might have been let to be scullery-maid\footnote{\textbf{scullery} n. a room that is near the kitchen in a large and usually old house and that is used for washing dishes, doing messy kitchen tasks, etc.} but I'd never have been let upstairs. I'm too common an' I talk too much Yorkshire. But this is a funny house for all it's so grand. Seems like there's neither Master nor Mistress except Mr. Pitcher an' Mrs. Medlock. Mr. Craven, he won't be troubled about anythin' when he's here, an' he's nearly always away. Mrs. Medlock gave me th' place out o' kindness. She told me she could never have done it if Misseltwaite had been like other big houses.''

``Are you going to be my servant?'' Mary asked, still in her imperious\footnote{\textbf{imperious} adj. having or showing the proud and unpleasant attitude of someone who gives orders and expects other people to obey them} little Indian way.

Martha began to rub her grate again.

``I'm Mrs. Medlock's servant,'' she said stoutly. ``An' she's Mr. Craven's--but I'm to do the house-maid's work up here an' wait on you a bit. But you won't need much waitin' on.''

``Who is going to dress me?'' demanded Mary.

Martha sat up on her heels again and stared. She spoke in broad Yorkshire in her amazement.

``Canna' tha' dress thysel'!'' she said.

``What do you mean? I don't understand your language,'' said Mary.

``Eh! I forgot,'' Martha said. ``Mrs. Medlock told me I'd have to be careful or you wouldn't know what I was sayin'. I mean can't you put on your own clothes?''

``No,'' answered Mary, quite indignantly\footnote{\textbf{indignant} adj. (\textbf{adv.} indignantly) feeling or showing anger because of something that is unfair or wrong}. ``I never did in my life. My Ayah dressed me, of course.''

``Well,'' said Martha, evidently not in the least aware that she was impudent, ``it's time tha' should learn. Tha' cannot begin younger. It'll do thee good to wait on thysel' a bit. My mother always said she couldn't see why grand people's children didn't turn out fair fools--what with nurses an' bein' washed an' dressed an' took out to walk as if they was puppies!''

``It is different in India,'' said Mistress Mary disdainfully. She could scarcely\footnote{\textbf{scarcely} adv. almost not at all} stand this.

But Martha was not at all crushed.

``Eh! I can see it's different,'' she answered almost sympathetically\footnote{\textbf{sympathetic} adj. (\textbf{adv.} sympathetically) feeling or showing concern about someone who is in a bad situation}. ``I dare say it's because there's such a lot o' blacks there instead o' respectable white people. When I heard you was comin' from India I thought you was a black too.''

Mary sat up in bed furious.

``What!'' she said. ``What! You thought I was a native. You--you daughter of a pig!''

Martha stared and looked hot.
